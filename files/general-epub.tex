\documentclass[ebook,12pt,oneside,openany]{memoir}
%\documentclass[oneside,onecolumn,openany,final]{book}
\usepackage{polyglossia}
\setmainlanguage{spanish}
\usepackage[autostyle=true]{csquotes}

\usepackage[style=authoryear,
backend=bibtex8,
sortcites=true,
indexing=cite,
backref=true,
indexing=cite]{biblatex}

% \defcounter{biburlnumpenalty}{3000}
% \defcounter{biburlucpenalty}{6000}
% \defcounter{biburllcpenalty}{9000}

% Generar los índices, desactivamos las opciones para títulos
% \renewbibmacro*{bibindex}{%
%   \ifbibindex
%     {\indexnames{author}%
%      \indexnames{editor}%
%      \indexnames{translator}%
%      \indexnames{commentator}}
%     {}}

% \renewbibmacro*{citeindex}{%
%   \ifciteindex
%     {\indexnames{author}%
%      \indexnames{editor}%
%      \indexnames{translator}%
%      \indexnames{commentator}}
%     {}}

% \usepackage[xindy]{imakeidx}
% \makeindex[name=names,title=Índice de autores]
% \makeindex
% \makeindex[name=concepto,title=Índice de conceptos]
% \makeindex[name=onomastico,title=Índice onomástico]

% \usepackage{esindex}
% \DeclareIndexNameFormat{default}{%
%     \usebibmacro{index:name}{\esindex[names]}
%     {\namepartfamily}
%     {\namepartgiven}
%     {\namepartprefix}
%     {\namepartsuffix}}
% \renewbibmacro*{citeindex}{%
%     \ifciteindex
%     {\indexnames{labelname}}
%     {}}

% Diseño de la raya del medio
\makeatletter
\def\thinskip{\hskip 0.16667em\relax}
\def\endash{--}
\def\emdash{\endash-}
\def\d@sh#1#2{\unskip#1\thinskip#2\thinskip\ignorespaces}
\def\dash{\d@sh\nobreak\endash}
\def\Dash{\d@sh\nobreak\emdash}
\def\ldash{\d@sh\empty{\hbox{\endash}\nobreak}}
\def\rdash{\d@sh\nobreak\endash}
\def\Ldash{\d@sh\empty{\hbox{\emdash}\nobreak}}
\def\Rdash{\d@sh\nobreak\emdash}
\def\hyph{-\penalty\z@\hskip\z@skip}
\def\slash{/\penalty\z@\hskip\z@skip}
\makeatother

\usepackage{froufrou}
\usepackage{graphicx}
\usepackage{booktabs}
\usepackage{svg}
\usepackage{xcolor}

% diseño de 3 asteriscos para los epub
\newcommand{\aste}{\begin{center}* * *\end{center}}

% configuración de los valores decimales de los cuadros
\usepackage{siunitx}
\sisetup{output-decimal-marker={.},
	group-separator={\,},
	group-minimum-digits=3,
	table-text-alignment=center,
	detect-all}

% diseño de listas
\usepackage{enumerate}

% generamos los glosarios
\usepackage[acronym,sanitizesort,nonumberlist]{glossaries}
\preto\chapter{\glsresetall}
\makenoidxglossaries
\renewcommand{\glsnamefont}[1]{\sf\textbf{\textup{#1}}}

\usepackage{endnotes,chngcntr}
\def\notesname{Notas}
\let\footnote=\endnote

\makeatletter
\renewcommand\enoteheading{%
    \setcounter{secnumdepth}{-2}
    \chapter*{\notesname\markboth{}{}}
    \mbox{}\par\vskip-\baselineskip
    \let\@afterindentfalse\@afterindenttrue
}
\makeatother

\usepackage{xparse}

\let\latexchapter\chapter

\RenewDocumentCommand{\chapter}{som}{%
    \IfBooleanTF{#1}
    {\latexchapter*{#3}}
    {\IfNoValueTF{#2}
        {\latexchapter{#3}}
        {\latexchapter[#2]{#3}}%
        \addtoendnotes{%
            \noexpand\enotedivision{\noexpand\subsection}
            {\thechapter \, \unexpanded{#3}}}% \chaptername\
    }%
}
\makeatletter
\def\enotedivision#1#2{\@ifnextchar\enotedivision{}{#1{#2}}}
\makeatletter

\usepackage{xurl}

\usepackage{ifthen}

% fin del preambulo
